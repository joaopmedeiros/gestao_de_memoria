\documentclass{article}

\usepackage[portuges]{babel}
\usepackage[latin1]{inputenc}
\usepackage{listings}

\parindent=0pt
\parskip=2pt

\title{Organiza��o e Arquitetura de Computadores - Trabalho 3 }
\author{Jo�o Paulo Medeiros Cecilio 
        \and 
        Bruno Bragan�a Mendes }
\date{Novembro de 2017}

\begin{document}

\maketitle



\section{Introdu��o}

\section{Background}
\lstset{language=Python}
\lstset{frame=lines}
\lstset{caption={Insert code directly in your document}}
\lstset{label={lst:code_direct}}
\lstset{basicstyle=\footnotesize}
\begin{lstlisting}
for i in lista_acessos:
    tag = i[0][:tam_tag]
    if tag in lista_mem_associativa:
        i[1] = 'HIT'
        dic_cache_controle_uso[tag] = dic_cache_controle_uso[tag] + 1
        cache[tag].add(i[0])
    else:
        i[1] = 'MISS'
        if(len(lista_mem_associativa)<tam_cache):
            lista_mem_associativa.append(tag)
            dic_cache_controle_uso[tag] = 1
            cache[tag] = set()
            cache[tag].add(i[0])
        else:
            for key in sorted(dic_cache_controle_uso):
                tag_menos_usada = key
                lista_mem_associativa[lista_mem_associativa.index(tag_menos_usada)] = tag
                dic_cache_controle_uso.pop(tag_menos_usada)
                dic_cache_controle_uso[tag] = 1
                cache.pop(tag_menos_usada)
                cache[tag] = set()
                cache[tag].add(i[0])
                break
\end{lstlisting}
\subsection{Conceitos b�sicos em XXX}
\subsubsection{Conceitos b�sicos em XXX.1}
Aqui vai um exemplo de uma lista numerada
\begin{enumerate}
\item primeira caracter�stica desta fase
\item segunda caracter�stica desta fase
\item terceira caracter�stica desta fase
\end{enumerate}

\subsubsection{Conceitos b�sicos em XXX.2}
Seguem-se algumas defini��es fundamentais para se perceberem as ideias 
defendidas a seguir:
\begin{description}
\item[conceito 1] respectiva defini��o 1
\item[conceito 2] descri��o do  conceito 2
\end{description}

\subsection{Conceitos b�sicos em YYY}

\section{A Proposta}

\section{A sua Implementa��o}

\section{Conclus�o}
S�ntese do que foi dito.\\
Lista dos resultados atingidos:
\begin{itemize}
\item resultado 1
\item resultado 2
\end{itemize}
Conclus�o final e Trabalho Futuro.
\end{document}
